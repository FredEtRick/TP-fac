%%%%%%%%%%%%%%%%%%%%%%%%%%%%%%%%%%%%%%%%%
% Modèle utilisé :"ModernCV" CV and Cover Letter
% Source :http://www.LaTeXTemplates.com
% Auteur original:Xavier Danaux (xdanaux@gmail.com)
%
% Important note:
% This template requires the moderncv.cls and .sty files to be in the same 
% directory as this .tex file. These files provide the resume style and themes 
% used for structuring the document.
%
%%%%%%%%%%%%%%%%%%%%%%%%%%%%%%%%%%%%%%%%%

%----------------------------------------------------------------------------------------
%	PACKAGES AND OTHER DOCUMENT CONFIGURATIONS
%----------------------------------------------------------------------------------------

\documentclass[11pt,a4paper,sans]{moderncv} % Font sizes: 10, 11, or 12; paper sizes: a4paper, letterpaper, a5paper, legalpaper, executivepaper or landscape; font families: sans or roman

\moderncvstyle{casual} % CV theme - options include: 'casual' (default), 'classic', 'oldstyle' and 'banking'
\moderncvcolor{blue} % CV color - options include: 'blue' (default), 'orange', 'green', 'red', 'purple', 'grey' and 'black'

\usepackage[francais]{babel}
\usepackage[T1]{fontenc}
\usepackage[utf8]{inputenc}


\usepackage{lipsum} % Used for inserting dummy 'Lorem ipsum' text into the template

\usepackage[scale=0.75]{geometry} % Reduce document margins
%\setlength{\hintscolumnwidth}{3cm} % Uncomment to change the width of the dates column
%\setlength{\makecvtitlenamewidth}{10cm} % For the 'classic' style, uncomment to adjust the width of the space allocated to your name

%----------------------------------------------------------------------------------------
%	NAME AND CONTACT INFORMATION SECTION
%----------------------------------------------------------------------------------------

\firstname{Quentin} % Your first name
\familyname{Philippot} % Your last name

% All information in this block is optional, comment out any lines you don't need
\title{Curriculum Vitae}
\address{18 rue de la lie}{Montpellier, 34090}
\mobile{0606060606}
\phone{0404040404}
\email{quentin.philippot@gmail.com}
%\extrainfo{additional information}
\photo[70pt][0.4pt]{pictures/picture} % The first bracket is the picture height, the second is the thickness of the frame around the picture (0pt for no frame)

%----------------------------------------------------------------------------------------

\begin{document}

%----------------------------------------------------------------------------------------
%	COVER LETTER
%----------------------------------------------------------------------------------------



\recipient{Département Informatique de la Faculté des Sciences}{Université Montpellier II\\Bt.16 - CC 12\\Place Eugène Bataillon, \\34095 Montpellier cedex 05} % Letter recipient
\date{\today} % Letter date
\opening{Madame, Monsieur,} % Opening greeting
\closing{Cordialement,} % Closing phrase
\enclosure[Joint]{curriculum vit\ae{}} % List of enclosed documents

\makelettertitle % Print letter title

%\lipsum[1-3] % Dummy text
Etudiant en troisième année de licence informatique à la faculté des science ( \textit{Université de Montpellier 2}), je me permets de vous adresser un dossier de candidature relative à mon entrée en Master 1 d’Informatique, sous votre responsabilité.
\newline \newline
Titulaire d'un baccalauréat à vocation scientique en double spécialité mathématique et physique, et après trois années consacrées à l'étude de l'informatique, cette démarche s'inscrit dans la continuité des formations déjà entreprises. L'informatique est en mon sens, un sujet de fascination conjugant sciences abstraites à un quotidien concret,  où technologies et efficacitées sont de mises. Poursuivre l'étude de cette discipline en master m'apparait comme une opportunité permettant d'approfondir mes connaissance dans ce domaine.
\newline \newline
Disposant de la capacité de travail et de la motivation nécessaires à la réussite d’un master, je
sollicite votre bienveillance à l’égard de mon dossier pour mon admission en Master 1 d’Informatique \textbf{MOCA}.
\newline \newline
Je vous prie d’agréer, Madame, Monsieur, l’expression de ma considération respectueuse.

\makeletterclosing % Print letter signature

%----------------------------------------------------------------------------------------

\end{document}
