%% start of file `template.tex'.
%% Copyright 2006-2013 Xavier Danaux (xdanaux@gmail.com).
%
% This work may be distributed and/or modified under the
% conditions of the LaTeX Project Public License version 1.3c,
% available at http://www.latex-project.org/lppl/.


\documentclass[11pt,a4paper,sans]{moderncv}        % possible options include font size ('10pt', '11pt' and '12pt'), paper size ('a4paper', 'letterpaper', 'a5paper', 'legalpaper', 'executivepaper' and 'landscape') and font family ('sans' and 'roman')

% moderncv themes
\moderncvstyle{banking}                            % style options are 'casual' (default), 'classic', 'oldstyle' and 'banking'
\moderncvcolor{blue}                                % color options 'blue' (default), 'orange', 'green', 'red', 'purple', 'grey' and 'black'
%\renewcommand{\familydefault}{\sfdefault}         % to set the default font; use '\sfdefault' for the default sans serif font, '\rmdefault' for the default roman one, or any tex font name
%\nopagenumbers{}                                  % uncomment to suppress automatic page numbering for CVs longer than one page

% character encoding
\usepackage[utf8]{inputenc}                       % if you are not using xelatex ou lualatex, replace by the encoding you are using
%\usepackage{CJKutf8}                              % if you need to use CJK to typeset your resume in Chinese, Japanese or Korean

% adjust the page margins
\usepackage[scale=0.75]{geometry}
%\setlength{\hintscolumnwidth}{3cm}                % if you want to change the width of the column with the dates
%\setlength{\makecvtitlenamewidth}{10cm}           % for the 'classic' style, if you want to force the width allocated to your name and avoid line breaks. be careful though, the length is normally calculated to avoid any overlap with your personal info; use this at your own typographical risks...

% personal data
\name{Jihen}{FOURATI}
\title{Candidature Master}                               % optional, remove / comment the line if not wanted
\address{570 Route de Ganges}{34090}{Montpellier}% optional, remove / comment the line if not wanted; the "postcode city" and and "country" arguments can be omitted or provided empty
\phone[mobile]{+33.(0)659262798}                   % optional, remove / comment the line if not wanted
%\phone[fixed]{}                    % optional, remove / comment the line if not wanted
%\phone[fax]{}                      % optional, remove / comment the line if not wanted
\email{jihen.fourati@etud.univ-montp2.fr}                               % optional, remove / comment the line if not wanted
\quote{Some quote}                                 % optional, remove / comment the line if not wanted

% to show numerical labels in the bibliography (default is to show no labels); only useful if you make citations in your resume
%\makeatletter
%\renewcommand*{\bibliographyitemlabel}{\@biblabel{\arabic{enumiv}}}
%\makeatother
%\renewcommand*{\bibliographyitemlabel}{[\arabic{enumiv}]}% CONSIDER REPLACING THE ABOVE BY THIS

% bibliography with mutiple entries
%\usepackage{multibib}
%\newcites{book,misc}{{Books},{Others}}
%----------------------------------------------------------------------------------
%            content
%----------------------------------------------------------------------------------
\begin{document}
%-----       letter       ---------------------------------------------------------
% recipient data
\recipient{Master Informatique IMAGINA}{Université Montpellier II\\2 Place Eugène Bataillon\\34095 Montpellier Cedex 5}
\date{\today}
\opening{Madame, Monsieur,}
\closing{Je vous prie d'agréer,
Madame/Monsieur, l'expression de
ma considération respectueuse.}
\enclosure[Attached]{curriculum vit\ae{}}          % use an optional argument to use a string other than "Enclosure", or redefine \enclname
\makelettertitle
Effectuant actuellement ma dernière année de Licence Informatique à l’Université Montpellier II, je me permets de vous adresser mon dossier de candidature pour le master IMAGINA que l’université de Montpellier II propose sous votre responsabilité.
Au cours de mes études universitaires, j’ai eu l’occasion de bénéficier d’un cursus fondé sur l'informatique, et, d’autre part ai travaillé dans des entreprises gérant des systèmes informatiques.
Particulièrement intéressé par l'image, j’ai profité de mon année universitaire pour approfondir mes connaissances dans ce domaine. J’ai notamment pu suivre un cours intitulé «Gestion de projet», dont l’objectif était d’apprendre, grâce différentes méthodes existantes, comment appréhender de nos jours l'environnement de travail en groupe, notamment en entreprise, et comment gérer de telles situations afin d’être en mesure de prendre les bonnes décisions. Cette expérience m’a aidé à développer mon sens de l'organisation, atout majeur dans le domaine de l'informatique.
Effectuer le master IMAGINA serait ainsi un complément essentiel à mes connaissances acquises et les enseignements dispensés s’inscriraient parfaitement dans la logique de mon projet professionnel.
Disposant de la capacité de travail et de la motivation nécessaires à la réussite d’un master, je sollicite votre bienveillance à l’égard de mon dossier pour mon admission au master IMAGINA.

\makeletterclosing

\end{document}


%% end of file `template.tex'.
