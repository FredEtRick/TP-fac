%% start of file `template.tex'.
%% Copyright 2006-2013 Xavier Danaux (xdanaux@gmail.com).
%
% This work may be distributed and/or modified under the
% conditions of the LaTeX Project Public License version 1.3c,
% available at http://www.latex-project.org/lppl/.


\documentclass[11pt,a4paper,sans]{moderncv}        % possible options include font size ('10pt', '11pt' and '12pt'), paper size ('a4paper', 'letterpaper', 'a5paper', 'legalpaper', 'executivepaper' and 'landscape') and font family ('sans' and 'roman')

% moderncv themes
\moderncvstyle{oldstyle}                           % style options are 'casual' (default), 'classic', 'oldstyle' and 'banking'
\moderncvcolor{grey}                               % color options 'blue' (default), 'orange', 'green', 'red', 'purple', 'grey' and 'black'
%\renewcommand{\familydefault}{\sfdefault}         % to set the default font; use '\sfdefault' for the default sans serif font, '\rmdefault' for the default roman one, or any tex font name
%\nopagenumbers{}                                  % uncomment to suppress automatic page numbering for CVs longer than one page

% character encoding
\usepackage[utf8]{inputenc}                       % if you are not using xelatex ou lualatex, replace by the encoding you are using
%\usepackage{CJKutf8}                              % if you need to use CJK to typeset your resume in Chinese, Japanese or Korean

% adjust the page margins
\usepackage[scale=0.75]{geometry}
%\setlength{\hintscolumnwidth}{3cm}                % if you want to change the width of the column with the dates
%\setlength{\makecvtitlenamewidth}{10cm}           % for the 'classic' style, if you want to force the width allocated to your name and avoid line breaks. be careful though, the length is normally calculated to avoid any overlap with your personal info; use this at your own typographical risks...

% personal data
\name{Julien}{Lemattre}
\title{Resumé title}                               % optional, remove / comment the line if not wanted
\address{470 chemin du Sablassou}{34170 Castelnau-le-lez}{France}% optional, remove / comment the line if not wanted; the "postcode city" and and "country" arguments can be omitted or provided empty
\phone[mobile]{+33~(6)~21~37~83~74}                   % optional, remove / comment the line if not wanted                     % optional, remove / comment the line if not wanted
\email{julien.lemattre@etud.univ-montp2.fr}                               % optional, remove / comment the line if not wanted
                         % optional, remove / comment the line if not wanted
                                % optional, remove / comment the line if not wanted

% to show numerical labels in the bibliography (default is to show no labels); only useful if you make citations in your resume
%\makeatletter
%\renewcommand*{\bibliographyitemlabel}{\@biblabel{\arabic{enumiv}}}
%\makeatother
%\renewcommand*{\bibliographyitemlabel}{[\arabic{enumiv}]}% CONSIDER REPLACING THE ABOVE BY THIS

% bibliography with mutiple entries
%\usepackage{multibib}
%\newcites{book,misc}{{Books},{Others}}
%----------------------------------------------------------------------------------
%            content
%----------------------------------------------------------------------------------
\begin{document}
%-----       letter       ---------------------------------------------------------
% recipient data
\recipient{Université Montpellier 2}{Département Informatique\\Place Eugène Bataillon\\Montpellier}
\date{13 Novembre 2014}
\opening{Madame, Monsieur}
\closing{Cordialement}
       % use an optional argument to use a string other than "Enclosure", or redefine \enclname
\makelettertitle

Je me permets de vous contacter afin de vous faire part de ma candidature pour l'année à venir, à l'entrée en première année de Master IMAGINA à l'Université Montpellier 2.
Je suis diplômé d'une licence en Informatique, obtenue dans cette même université.
Ce Master m'intéresse tout particulièrement, car je souhaite apprendre les techniques de développement des jeux video.
Je m'investis particulièrement dans ce domaine depuis longtemps car il constitue ma passion et mon objectif professionnel. J'ai déjà pris part à des Game Jam à Monpellier, ce qui témoigne de mon goût pour la création de jeux. De plus, je suis organisateur d'un festival centré sur le jeu video qui aura lieu à Fabrègues les 24 et 25 Janvier prochains, nommé le Land of Games.
Mon parcours est atypique, puisque je suis entré à l'Université de Sciences Montpellier 2 après un bac Littéraire. Cependant, voulant étudier l'informatique, j'ai travaillé seul pour ratrapper le programme de bac Scientifique en Mathématiques ou Physique, afin de pouvoir suivre ce cursus.

En espérant une réponse favorable de votre part, je vous prie d'agréer le sentiment de mes salutations distinguées.



\makeletterclosing

\end{document}


%% end of file `template.tex'.
