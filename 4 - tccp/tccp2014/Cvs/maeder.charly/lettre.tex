\documentclass[11pt,origdate]{lettre}
\usepackage{palatino}
\usepackage[T1]{fontenc}
\usepackage[latin1]{inputenc}
\usepackage[frenchb]{babel}

\begin{document}
	\begin{letter}{Monsieur le directeur des masters informatiques\\ % Destinataire
		           rue de la fac des sciences\\
		           34000 Montpellier} 
		\address{Charly Maeder\\ % Mon adresse
		         rue du g\'en\'eral Vincent\\
				 34000 Montpellier}
		\name{Charly Maeder} % Mon nom
		%\location{Mon d\'epartement dans lentreprise}
		\telephone{06 -- -- -- --}
		\email{charly.maeder@live.fr}
		\nofax % pas de fax. Alternative: \fax{numero}
		
		\francais % Met les labels en français et le \closing{} en pleine largeur
		          % Variantes: \anglais, \americain, et \allemand
		\pagestyle{empty} % alternatives : plain (num\'ero de page en pied), headings (entête avec lieu et date)
		
		\conc{\textbf{Candidature au Master}}
		\lieu{Montpellier} % Lendroit doù j\'ecris
		
		\renewcommand{\concname}{} % Elimine laffichage du texte "Objet :"
		\renewcommand{\emaillabelname}{} % Elimine laffichage du texte "E-mail :"
		
		%\def\openingspace{10mm} % ajuste lespace vertical autour du champ sujet, default: 1cm
 		\def\sigspace{10mm} % Espacement vertical entre texte et signature(s), default: 1.5cm
		
		\makeatletter
		% BOITE DENTETE
		\def\pict@let@width{185}       % default: 185
		\def\pict@let@height{65}       % default: 65
		\def\pict@let@hoffset{0}       % default: 0
		\def\pict@let@voffset{0}       % default: 0
		% TRAIT DE PLIAGE
		\def\rule@hpos{-25}            % default: -25
		\def\rule@vpos{-15}            % default: -15
		\def\rule@length{10}           % default: 10
		% ADRESSE DE LEXPEDITEUR
		\def\fromaddress@let@hpos{-10} % default: -10
		\def\fromaddress@let@vpos{80}  % default: 70
		                               % je remonte l\'egèrement ladresse vers le coin haut gauche
		\fromaddress@let@width=69mm    % default: 69
		% LIEU DEXPEDITION
		\def\fromlieu@let@hpos{90}     % default: 90
		\def\fromlieu@let@vpos{0}      % default: 62 
		                               % je d\'eplace lieu et date sous ladresse du destinataire
		\fromlieu@let@width=65mm       % default: 69
		% ADRESSE DU DESTINATAIRE
		\def\toaddress@let@hpos{90}    % default: 90
		\def\toaddress@let@vpos{40}    % default: 40
		\toaddress@let@width=80mm      % default: 80
		\makeatother

		% DEBUT DE LA LETTRE
		\opening{Madame, Monsieur, }
		Effectuant actuellement mon ann\'ee de Licence d'informatique \`{a} l'Universit\'e de Montpellier 2 \`{a} Montpellier, je me permets de vous adresser mon dossier de candidature pour le Master de XXXXX que l'universit\'e de Montpellier 2 propose sous votre responsabilit\'e.

		Au cours de mes \'etudes universitaires, j'ai eu l'occasion de b\'en\'eficier d'un cursus fond\'e en informatique.

		Particuli\`erement int\'eress\'e par XXXX, j'ai profit\'e de mon ann\'ee universitaire \`{a} l'UM2 pour approfondir mes connaissances dans ce domaine.

		Effectuer le Master XXXX serait ainsi un compl\'ement essentiel \`{a} mes connaissances acquises et les enseignements dispens\'es sinscriraient parfaitement dans la logique de mon projet professionnel. Mon objectif est de travailler dans le service XXXX d'une grande entreprise.

		Disposant de la capacit\'e de travail et de la motivation n\'ecessaires \ \`{a} la r\'eussite d'un master, je sollicite votre bienveillance \`{a} l'\'egard de mon dossier pour mon admission au Master XXXXX.

		Je vous prie d'agr\'eer, Madame ou Monsieur, l'expression de ma consid\'eration respectueuse.
		\closing{Veuillez croire, Madame, aux sentiments les plus sinc\`eres.}
		
	\end{letter}
\end{document}